\documentclass[11pt,a4paper]{article}
\usepackage[spanish,es-nodecimaldot]{babel}	% Utilizar español
\usepackage[utf8]{inputenc}					% Caracteres UTF-8
\usepackage{graphicx}						% Imagenes
\usepackage[hidelinks]{hyperref}			% Poner enlaces sin marcarlos en rojo
\usepackage{fancyhdr}						% Modificar encabezados y pies de pagina
\usepackage{float}							% Insertar figuras
\usepackage[textwidth=397pt]{geometry}		% Anchura de la pagina
\usepackage[nottoc]{tocbibind}				% Referencias (no incluir num pagina indice en Indice)
\usepackage{enumitem}						% Permitir enumerate con distintos simbolos
\usepackage[T1]{fontenc}					% Usar textsc en sections
\usepackage{amsmath}						% Símbolos matemáticos
\usepackage{listings}
\usepackage{algorithm}

\usepackage{color}
 
\definecolor{codegreen}{rgb}{0,0.6,0}
\definecolor{codegray}{rgb}{0.5,0.5,0.5}
\definecolor{codepurple}{rgb}{0.58,0,0.82}
\definecolor{backcolour}{rgb}{0.95,0.95,0.92}
 
\lstdefinestyle{mystyle}{
    backgroundcolor=\color{backcolour},   
    commentstyle=\color{codegreen},
    keywordstyle=\color{magenta},
    numberstyle=\tiny\color{codegray},
    stringstyle=\color{codepurple},
    basicstyle=\footnotesize,
    breakatwhitespace=false,         
    breaklines=true,                 
    captionpos=b,                    
    keepspaces=true,                 
    numbers=left,                    
    numbersep=5pt,                  
    showspaces=false,                
    showstringspaces=false,
    showtabs=false,                  
    tabsize=2
}
 
\lstset{style=mystyle, language=Lisp}



% Comando para poner el nombre de la asignatura
\newcommand{\asignatura}{Ingeniería del Conocimiento}
\newcommand{\autor}{Vladislav Nikolov Vasilev}

% Configuracion de encabezados y pies de pagina
\pagestyle{fancy}
\lhead{\autor{}}
\rhead{\asignatura{}}
\lfoot{Grado en Ingeniería Informática}
\cfoot{}
\rfoot{\thepage}
\renewcommand{\headrulewidth}{0.4pt}		% Linea cabeza de pagina
\renewcommand{\footrulewidth}{0.4pt}		% Linea pie de pagina

\begin{document}
\pagenumbering{gobble}

% Pagina de titulo
\begin{titlepage}

\begin{minipage}{\textwidth}

\centering

\includegraphics[scale=0.5]{img/ugr.png}\\

\textsc{\Large \asignatura{}\\[0.2cm]}
\textsc{GRADO EN INGENIERÍA INFORMÁTICA}\\[1cm]

\noindent\rule[-1ex]{\textwidth}{1pt}\\[1.5ex]
\textsc{{\Huge PRÁCTICA FINAL\\[0.5ex]}}
\textsc{{\Large SBC para el riego automático de cultivos y plantas\\}}
\noindent\rule[-1ex]{\textwidth}{2pt}\\[3.5ex]

\end{minipage}

\vspace{0.5cm}

\begin{minipage}{\textwidth}

\centering

\textbf{Autor}\\ {\autor{}}\\[1ex]
\textbf{DNI}\\ {X8743846M}\\[1ex]
\textbf{Rama}\\ {Computación y Sistemas Inteligentes}\\[1ex]
\vspace{0.2cm}

\includegraphics[scale=0.3]{img/etsiit.jpeg}

\vspace{0.5cm}
\textsc{Escuela Técnica Superior de Ingenierías Informática y de Telecomunicación}\\
\vspace{0.5cm}
\textsc{Curso 2018-2019}
\end{minipage}
\end{titlepage}

\pagenumbering{arabic}
\tableofcontents
\thispagestyle{empty}				% No usar estilo en la pagina de indice

\newpage

\setlength{\parskip}{1em}

\section{Resumen del funcionamiento del sistema}

El sistema ha sido implementado utilizando \textbf{CLIPS} y \textbf{Python3}, más concretamente la versión 3.7.1 de Python.
Para poder utilizar el sistema, se recomienda tener instalada una versión de Python igual o superior a la 3.7.1,
además de tener instalados en el equipo los módulos \textit{numpy} \cite{bib:numpy} y \textit{scikit-fuzzy}
\cite{bib:skfuzzy}, los cuáles son utilizados por el programa.

El sistema implementado en CLIPS es el sistema experto principal. Éste utiliza un pequeño programa escrito en Python que se encarga
de realizar las deducciones correspondientes al riego de las plantas mediante un sistema basado en \textit{lógica difusa}.
Los dos programas se comunican enviándose la información necesaria para realizar las deducciones y las deducciones mismas mediante
archivos.

El sistema principal se encuentra en \textit{sistema.clp}, mientras que \textit{sensores-init.clp} contiene información inicial
relativa a los sensores (para no tener que insertar unos datos iniciales a mano) y \textit{previsiones.clp} contiene una serie
de previsiones del tiempo que se recomienda encarecidamente cargar en el sistema para que pueda funcionar de forma correcta,
ya que el sistema supone que contiene información completa sobre la previsión del tiempo para las siguientes horasr.
Estas previsiones pueden ser modificadas, tal y como ser indicará posteriormente. El archivo \textit{fuzzy.py} contiene
el sistema basado en lógica difusa que se encarga de realizar las deducciones para el riego de las plantas.

En general, el funcionamiento del sistema es bastante simple. El sistema experto va recibiendo valores de entrada de los sensores
relativos a la \textbf{humedad}, \textbf{temperatura} y \textbf{luminosidad} de las plantas, además de valores de predicciones del
tiempo.
Con estos datos, el sistema decide si activar el \textbf{vaporizador} en aquellas plantas que considere oportunas con tal de
reducir su temperatura o si activar el sistema de \textbf{riego} de las plantas, utilizando para ello los valores de los sensores.
El sistema experto principal le envía los datos al sistema difuso para que se encargue de decidir cuál es el mejor curso de acción.
Una vez decidido, envía los resultados de vuelta, y el sistema experto manipula esta información utilizando las predicciones de las
próximas 8 horas con el objetivo de determinar cuál es el mejor curso de acción a tomar.

El sistema también permite pasar las horas una a una, simulando el paso del tiempo y actualizando el conocimiento de forma
correspondiente, deduciendo por ejemplo si en la hora actual está lloviendo según la previsión que había para esa hora.

\newpage

\section{Descripción del proceso de creación del sistema}

\subsection{Procedimiento seguido para el desarrollo de la base de conocimiento}

Para el desarrollo del sistema, se han ido satisfaciendo los requisitos especificados uno a uno. Estos requisitos son los
siguientes:

\begin{enumerate}
	\item Ser consciente de las necesidades de humedad de cada cultivo. Se ha de considerar, al menos, tres tiestos con rangos
	diferentes de humedad ideal.
	\item Activar el riego automático de cada tiesto independientemente cuando dicho cultivo lo necesite.
	\item Desactivar el riego de cada tiesto cuando se llegue a la humedad deseada.
	\item Realizar un riego más inteligente en función del momento del día, considerando la posible evaporación.
	\item Evitar, siempre que se pueda, el riego en situaciones de altas temperaturas y sol. No se podrá posponer en presencia de
	valores de humedad críticamente bajos.
	\item Activar vaporizadores para refrescar el ambiente de cada tiesto y reducir altas temperaturas.
	\item Posibilitar la combinación del uso de vaporizadores con el riego en momentos de altas temperaturas, cuando sea
	estrictamente necesario.
	\item Utilizar la información meteorológica de la zona local para gestionar el riego de manera más eficiente, como posponer o
	regar con menor intensidad en caso de lluvia, según su intensidad y la certeza de la predicción.
\end{enumerate}

Primero se ha comenzado especificando las plantas a utilizar, que en este caso han sido un \textbf{cactus}, un \textbf{tulipán}
y una \textbf{palmera}, y se han especificado los valores mínimos y máximos de humedad, temperatura y luminosidad, además de los
valores críticos de humedad y temperatura.

Sobre estos datos se ha añadido la capacidad de activar y desactivar el riego de forma automática para cada planta según
su humedad. Posteriormente, se introdujo el sistema difuso escrito en Python3 que se encargaba de determinar si regar las plantas
o no y con qué intensidad se deben regar en función de la información proporcionada por los sensores. Estas salidas se verán
más adelante. Además de eso, se explicarán qué reglas ha seguido el sistema difuso para obtener las salidas.

Posteriormente, se añadió la capacidad al sistema de activar y desactivar los vaporizadores de las plantas, tal y cómo se hacía con
el sistema de riego. Los vaporizadores tienen precedencia al riego, ya que si se usan antes de regar para reducir las temperaturas,
se pueden conseguir unos riegos más eficaces, ya que se podrá regar más.

Finalmente, para poder trabajar con el tiempo atmosférico, se introdujo en el sistema la capacidad de representar el tiempo y las
predicciones de tiempo según la hora, además del tiempo actual (si llueve o no y con qué intensidad lo hace). También se modifició
el funcionamiento del sistema para poder manipular las salidas que le proporcionaba el sistema difuso para que, según el pronóstico
de lluvias de las próximas 8 horas, pueda ser capaz de deducir si se debe regar menos de lo que indica la salida o si se debería
proceder tal y como ha indicado el sistema difuso. En secciones posteriores veremos cómo se han representado estos hechos y cómo
se han ido deduciendo las cosas en el sistema.

\subsection{Procedimiento de validación y verificación del sistema}

La validación y verificación del sistema ha sido un proceso que se ha ido haciendo a lo largo de todo el desarrollo, desde las
etapas iniciales hasta las finales.

\subsubsection{Verificación}

En cuanto a la verificación, se ha comprobado que el sistema fuese el correcto en cada etapa. A medida que se iba añadiendo
funcionalidad al sistema se iba comprobando que ésta funcionase correctamente. Por ejemplo, si se incluía una nueva regla, se 
probaba de forma interactiva añadiendo hechos a la base de conocimiento con el objetivo de ver si esa regla nueva era capaz de
activarse o no.
También se ha ido comprobando a medida que se iba construyendo el sistema que no hubiesen \textbf{errores sintácticos} ni
\textbf{inconsistencias}, como por ejemplo reglas que nunca se llegasen a disparar o que produjesen resultados contradictorios
entre ellas. También se ha comprobado que no se produjesen inconsistencias semánticas con el fin de mantener la coherencia
dentro del sistema. Se ha ido comprobando, por ejemplo, si los valores deducidos se correspondían con el tipo que debían tener.

Otra cosa que se ha hecho para validar el sistema es simular el comportamiento de un usuario para ver que podía trabajar con
normalidad sin que se produjesen errores o inconsistencias.

Con todo el proceso de verificación se cree que se han eliminado la mayoría o casi todos los errores, ya que se ha ido haciendo de
forma progresiva. Sin embargo, nunca podemos estar seguros del todo, con lo cuál es posible que aún exista algún que otro pequeño
error que no hayamos podido detectar debido a que, por motivos de tiempo, es imposible simular toda la actividad que realizaría
un usuario, ya que existen infinitos casos que se pueden dar.

\subsubsection{Validación}

En cuanto a la validación, se ha ido comprobando en cada fase que se cumplían los requisitos especificados. Además, se ha comprobado
que los resultados obtenidos fuesen correctos según las entradas que se le iba proporcionando al sistema. Aparte de esto, se
ha comprobado que la comunicación entre el sistema principal y el sistema difuso fuese correcta, que fuese un sistema relativamente
rápido y que fuese capaz de funcionar en tiempo real y que, además de todo esto, el sistema explicase suficientemente bien todo lo
que sucede en el sistema, explicando al usuario todas las deducciones que va haciendo de tal forma que éste tenga en todo momento
suficiente información para saber qué es lo que está ocurriendo.

Se cree que, en general, los resultados que ofrece el sistema son buenos. Sin embargo, al no haber extraído el conocimiento de
un experto si no de nosotros mismos, ya que nos hemos considerado los expertos, puede que hayamos cometido algún error a la hora de,
por ejemplo, especificar con qué intensidad se debe regar una planta en función de la información proporcionada por los sensores
de entrada. Esto es inevitable, ya que no tenemos una fuente real de conocimiento experto. Sin embargo, el sistema está hecho de tal
forma que se pueda modificar el funcionamiento fácilmente.

Por ejemplo, si no se está de acuerdo con algún resultado obtenido de, por ejemplo, el riego que se debe realizar, se puede revisar
el sistema difuso y modificar las reglas del riego para que produzcan unos resultados que se ajusten más a los que deberían
obtenerse realmente o a los deseados por el usuario.

\newpage

\section{Descripción del sistema}

\subsection{Variables de entrada del problema}
\label{subsect:input}

Las entradas al sistema son los valores que miden los sensores de humedad, temperatura y luminosidad para cada planta y la previsión
de lluvia para una determinada hora.

Las entradas de los sensores tienen la siguiente forma:

\begin{lstlisting}
(valor Tipo Planta Valor-Medido)
\end{lstlisting}

\noindent donde:

\begin{itemize}
	\item \textit{Tipo} representa Humedad, Temperatura o Luminosidad.
	\item \textit{Planta} hace referencia a una de las tres plantas (Cactus, Tulipan o Palmera).
	\item \textit{Valor-Medido} es el valor medido por el sensor. Todos son valores enteros y toman los siguients valores:
	
	\begin{itemize}
		\item En el caso de la \textbf{humedad}, los valores posibles están en el rango $[0, 1023]$, donde 0 indica humedad máxima
		y 1023 indica lo más seco posible.
		\item En el caso de \textbf{temperatura}, los valores posibles son todos los enteros, tanto positivos como negativos.
		Sin embargo, el sistema difuso solo acepta valores en el rango $[0, 99]$, ya que no parece muy posible que las plantas
		que se cuiden sobrevivan a temperaturas superiores o inferiores a esas.
		\item En el caso de \textbf{luminosidad}, los valores van desde 0 a cualquier valor, no hay límite. Sin embargo, el
		sistema difuso asume que la luminosidad más grande es 1000. De ser necesario, se puede ampliar.
	\end{itemize}
\end{itemize}

Las entradas de las previsiones tienen la siguiente forma:

\begin{lstlisting}
(prevision_lluvia Hora Intensidad)
\end{lstlisting}

\noindent donde:

\begin{itemize}
	\item \textit{Hora} es un valor entero en el rango $[0, 23]$ que representa a qué hora hace referencia la previsión.
	\item \textit{Intensidad} es un valor real que indica cuál es la intensidad de la lluvia en mm/h.
\end{itemize}

Estos datos de entrada son procesados para facilitarle el funcionamiento al sistema posteriormente. Para cada \textbf{valor}
de entrada, se generan estos dos hechos:

\begin{lstlisting}
(valor_registrado Tiempo Tipo Planta Valor-Medido)
(utlimo_registro Tipo Planta Tiempo)
\end{lstlisting}

\noindent donde \textit{Tiempo} es un valor que representa un tiempo desde que se inició el sistema. Esta es una forma
de identificar a los valores registrados y hacerlos únicos. Así, desde el \textbf{ultimo\_registro} se puede acceder fácilmente
al \textit{Valor-Medido} utilizando \textit{Tiempo}.

Para las previsiones, se obtienen los siguientes datos de entrada procesados:

\begin{lstlisting}
(LluviaPrevista Hora Tipo Intensidad)
\end{lstlisting}

\noindent donde \textit{Tipo} va en función de la intensidad. Es decir:

\begin{itemize}
	\item Si \textit{Tipo} es \textbf{no} es que no va a llover, lo cuál significa que \textit{Intensidad} está en el
	rango $[0, 0.2)$.
	\item Si \textit{Tipo} es \textbf{debil} es que habrá lluvia de intensidad débil, lo cuál significa que
	\textit{Intensidad} está en el rango $[0.2, 6.5)$.
	\item Si \textit{Tipo} es \textbf{moderada} es que habrá lluvia de intensidad moderada, lo cuál significa que
	\textit{Intensidad} está en el rango $[6.5, 15)$.
	\item Si \textit{Tipo} es \textbf{fuerte} es que habrá lluvia de intensidad fuerte, lo cuál significa que
	\textit{Intensidad} está en el rango $[15, 100)$.
	\item Si \textit{Tipo} es \textbf{torrencial} es que habrá lluvia de intensidad torrencial, lo cuál significa que
	\textit{Intensidad} está en el rango $[100, \infty)$.
\end{itemize}

\subsection{Variables de salida del problema}

Como tal, el sistema no produce ningún valor de salida. Sin embargo, tiene que interactuar con los sistemas de riego y vaporización.
Para ello, el sistema tiene que activar/desactivar estos sistemas, con lo cuál usará una serie de hechos.

Para el caso del riego, se tiene el siguiente hecho:

\begin{lstlisting}
(regar Planta Estado [HumedadObjetivo])
\end{lstlisting}

\noindent donde se tiene que:

\begin{itemize}
	\item \textit{Planta} es la planta asociada a dicho sistema de riego.
	\item \textit{Estado} indica si el sistema de riego está encendido o apagado mediante los valores on/off.
	\item \textit{HumedadObjetivo} es un valor que solo aparece si \textit{Estado} está en on. Este valor indica cuál es la humedad
	que se quiere conseguir con el riego.
\end{itemize}

En cuanto a la vaporización, tenemos el siguiente hecho:

\begin{lstlisting}
(vaporizador Planta Estado [TemperaturaObjetivo])
\end{lstlisting}

\noindent donde se tenemos:

\begin{itemize}
	\item \textit{Planta} es la planta asociada a dicho sistema de vaporización.
	\item \textit{Estado} indica si el sistema de vaporización está encendido o apagado mediante los valores on/off.
	\item \textit{TemperaturaObjetivo} es un valor que solo aparece si \textit{Estado} está en on. Este valor indica cuál es la
	temperatura a conseguir con la vaporización.
\end{itemize}

\subsection{Conocimiento global del sistema}

El sistema tiene una serie de hechos que son gloables a todo el sistema y que se cargan al inicio. A continuación se van a describir
brevemente cuáles son estos hechos, explicando como se llaman y la funcionalidad que tienen:

\begin{itemize}
	\item \textbf{Tiesto}: Contiene los tipos de plantas especificados. Inicialmente hay 3, tal y como se mencionó anteriormente:
	un cactus, un tilipán y una palmera.
	\item \textbf{HumedadMin}, \textbf{HumedadMax} y \textbf{HumedadCrit}: Valores enteros que representan las humedades mínima,
	máxima y crítica de la planta. Esto significa que la humedad ideal o media de la planta planta está entre el mínimo y el máximo.
	Si está por encima del máximo será una humedad alta, y si está por debajo del mínimo será baja. Adicionalemnte, si la humedad
	es muy próxima al valor crítico o baja de este, significará que obligatoriamente se tiene que regar la planta\footnote{Si se
	desean modificar estos valores, se le da la libertad al usuario para hacerlo. El sistema difuso funciona con estos conceptos
	de humedad por intervalos. Si se desean modificar los valores que permiten crear los conjuntos difusos, se tiene la libertad
	de modificar el programa en Python3 que contiene los valores para crear estos conjuntos.}. Un ejemplo de este conocimiento es
	el siguiente hecho que representa un valor de HumedadMin (para los otros casos solo cambia el primer valor, pero la estructura
	es la misma):
\begin{lstlisting}
	(HumedadMin Cactus 800)
\end{lstlisting}
	\item \textbf{TemperaturaMin}, \textbf{TemperaturaMax} y \textbf{TemperaturaCrit}: Valores enteros que representan la temperatura
	mínima, máxima y crítica de la planta, respectivamente. Al igual que en el caso anterior, se considerará que la temperatura
	ideal o media se encuentra entre los valores mínimo y máximo. Una temperatura baja se encontrará por debajo del mínimo y una
	alta por encima del máximo\footnote{Tal y como se dijo antes, estos valores se pueden modificar. El sistema difuso también
	utiliza rangos para las temperaturas para definir los conjuntos difusos, así que dichos valores se pueden modificar de ser
	necesario.}. La temperatura crítica indica un valor de temperatura alta que ya no puede ser tolerado por la
	planta, y que por tanto, debe ser \textbf{vaporizada} si se excede este valor. A continuación se muestra un ejemplo que
	representa la temperatura mínima, aunque para los otros hechos la estructura es la misma:
\begin{lstlisting}
(TemperaturaMin Cactus 15)
\end{lstlisting}
	\item \textbf{LuminosidadMin} y \textbf{LuminosidadMax}: Valores enteros que representan la luminosidad mínima y máxima de
	una planta. Como antes, la luminosidad ideal o media se encuentra entre los valores máximos y mínimos. Por debajo del mínimo
	se considera que hay una baja luminosidad, mientras que por encima del máximo que hay una alta luminosidad \footnote{De nuevo,
	estos valores se pueden modificar. Los valores que utiliza el sistema difuso para crear los conjuntos difusos también se pueden
	modificar para adaptarlos a las necesidades.}.	La luminosidad será una especie
	de indicativo de la cantidad de luz que recibe la planta, y por ende, es una representación del momento del día en el que nos
	encontramos. La luminosidad influirá a la hora de regar más o menos las plantas. A continuación se muestra un ejemplo que
	representa la luminosidad mínima, y para el caso de la máxima, la estructura es la misma solo que cambia el primer valor:
\begin{lstlisting}
(LuminosidadMin Cactus 300)
\end{lstlisting}
	\item \textbf{Regado}: Valor entero que indica cuánto incrementa la humedad de la planta por cada unidad de tiempo que se pasa
	regando. Se representa de la siguiente forma (el valor asociado puede cambiarse):
\begin{lstlisting}
(Regado 100)
\end{lstlisting}
	\item \textbf{Vaporizado}: Valor entero que indica cuánto se decrementa la temperatura de la planta por cada unidad de tiempo
	que se pasa vaporizando. Se representa de la siguiente forma (el valor puede cambiar):
\begin{lstlisting}
(Vaporizado 2)
\end{lstlisting}
	\item \textbf{hora}: Valor entero que representa la hora actual. Toma valores en el rango $[0, 23]$. Inicialmente se indica que
	son las 8 de la mañana:
\begin{lstlisting}
(hora 8)
\end{lstlisting}
	\item \textbf{LluviaActual}: Hecho que representa si actualmente está lloviendo o no, dependiendo de la previsión que había
	para esa hora en concreto. Inicialmente se indica que para la hora actual no está lloviendo. Los posibles valores de LLuviaActual
	que se pueden tener son \textbf{no}, \textbf{debil}, \textbf{moderada}, \textbf{fuerte} y \textbf{torrencial}. A continuación
	se muestra el valor inicial, tal y como se ha mencionado anteriormente:
\begin{lstlisting}
(LluviaActual no)
\end{lstlisting}
\end{itemize}

Adicionalmente, tal y como se indicó anteriormente, hay dos archivos extra, \textit{sensores-init.clp} y \textit{previsiones.clp}
que contienen hechos iniciales relacionados con los sensores de los tiestos y las previsiones del tiempo. Estos valores siguien
la misma estructura que los datos de entrada mostrados en la sección \ref{subsect:input}. Se recomienda cargar estos hechos, ya
que evitan tener que introducirlos a mano. Este conocimiento inicial se ha separado del resto del sistema para que sea más fácil
localizarlo y modificarlo y añadirle lo que se necesite de ser necesario. Además, puede que el usuario no quiera cargar estos
hechos iniciales y que los quiera definir él mismo.

\subsection{Especificación de los módulos desarrollados}


El sistema experto utiliza 3 módulos distintos. Aparte de ellos, el sistema tiene una serie de reglas que pueden funcionar
independientemente del módulo activo, como por ejemplo leer las entradas de los sensores o actualizar la hora y las previsiones
de tiempo.

A continuación se describe cada uno de los módulos:

\begin{itemize}
	\item \textbf{Módulo Riego}: Este módulo se encarga de decidir el riego de las plantas utilizando las entradas de los sensores,
	las previsiones de tiempo y los resultados proporcionados por el sistema difuso. Cada vez que llega una nueva entrada a
	cualquiera de los sensores o se cambia la hora se comprueba si se puede regar dicha planta o no. El módulo utiliza el resultado
	obtenido por el sistema difuso para saber si debe regar o no y se lo comunica al usuario. En caso de decidir regar, activa
	los sistemas de riego de la planta y se encarga de llegar a la humedad ideal. Antes de utilizar este módulo se activa un módulo
	que veremos un poco más adelante que se encarga de deducir si habrá alguna lluvia en las próximas 8 horas con el objetivo de
	modificar los resultados que ha proporcionado el sistema difuso.
	\item \textbf{Módulo Vaporizacion}: Este módulo se activa cuando el sistema ha decidido realizar la vaporización de una planta
	debido a las altas temperaturas que se han detectado. Una vez activado, se encarga de la vaporización de la planta hasta que
	los sensores de temperatura detecten que se ha llegado al valor especificado.
	\item \textbf{Módulo CalculoSiguienteLluvia}: Este módulo es utilizado justo antes de utilizar el módulo \textbf{Riego} y se
	encarga de deducir si en las próximas 8 horas se va a producir una lluvia, o si en el caso contrario, no se producirá. Para
	ello, utiliza información referente a las previsiones de tiempo. Para que el módulo funcione correctamente, se tienen que
	tener las previsiones para todas las horas, ya que se asume que se tiene información completa sobre el tiempo, aunque esta pueda
	cambiar posteriormente.
\end{itemize}

El sistema va activando y desactivando uno u otro módulo en función de lo que vaya deduciendo. Si decide regar es que ha deducido
que es necesario realizar un riego, y por tanto primero activaría el módulo \textbf{CalculoSiguienteLluvia} que se encarga de
deducir si va a llover o no y posteriormente activaría el módulo \textbf{Riego} para decidir si debe o no regar la planta con la
información de la que dispone. Si por ejemplo deduce que debe vaporizar una planta, el sistema activará el módulo
\textbf{Vaporizacion} para realizar la vaporización de la planta, y una vez que ha terminado, la desactivaría.

Es importante destacar que la vaporización, tal y como se mencionó anteriormente, tiene una mayor prioridad que el riego. Es decir,
el sistema dará preferencia a la deducción de la necesidad de vaporizar a la de regar, y por tanto, activará antes el módulo
\textbf{Vaporizacion} que el módulo \textbf{Riego}.

\subsection{Hechos y reglas de cada módulo}

A continuación se van a presentar de forma superficial las reglas que utiliza cáda módulo y se va a explicar muy brevemente
su funcionamiento. Si se quiere obtener más información sobre el funcionamiento concreto de las reglas, se recomienda consultar
la implementación del sistema.

Antes de ver las reglas, vamos a mostrar cuáles son las reglas que utiliza el sistema difuso para decidir si debe o no realizar
el riego. Se recuerda que no somos expertos en el riego de las plantas, pero que las reglas representan lo que haríamos:

\begin{gather*}
	\text{si} \; \; Humedad(Baja) \land Luminosidad(Baja) \land Temperatura(Baja) \rightarrow \; \; \text{regar media} \\
	\text{si} \; \; Humedad(Baja) \land Luminosidad(Baja) \land Temperatura(Media) \rightarrow \; \; \text{regar alta} \\
	\text{si} \; \; Humedad(Baja) \land Luminosidad(Baja) \land Temperatura(Alta) \rightarrow \; \; \text{regar media} \\
	\text{si} \; \; Humedad(Baja) \land Luminosidad(Media) \land Temperatura(Baja) \rightarrow \; \; \text{regar media} \\
	\text{si} \; \; Humedad(Baja) \land Luminosidad(Media) \land Temperatura(Media) \rightarrow \; \; \text{regar alta} \\
	\text{si} \; \; Humedad(Baja) \land Luminosidad(Media) \land Temperatura(Alta) \rightarrow \; \; \text{regar poca} \\
	\text{si} \; \; Humedad(Baja) \land Luminosidad(Alta) \land Temperatura(Baja) \rightarrow \; \; \text{regar poca} \\
	\text{si} \; \; Humedad(Baja) \land Luminosidad(Alta) \land Temperatura(Media) \rightarrow \; \; \text{regar media} \\
	\text{si} \; \; Humedad(Baja) \land Luminosidad(Alta) \land Temperatura(Alta) \rightarrow \; \; \text{regar poca} \\
	\text{si} \; \; Humedad(Media) \lor Humedad(Alta) \rightarrow \; \; \text{no regar}
\end{gather*}

Las salidas que devuelve son con qué intensidad regar en los casos en los que se debe hacer (\textbf{poca}, \textbf{media} o
\textbf{alta} intensidad) o si no debe hacerse. Poca intensidad significa regar la planta hasta alcanzar una humedad un poco
superior al mínimo. Intensidad media significa regar las plantas para conseguir una humedad un poco por debajo del máximo. Alta
intensidad significa regar las plantas para conseguir una humedad un poco superior a la del máximo.

Con esto dicho, vamos a ver cómo serían las reglas del sistema.

\subsubsection{Reglas generales}

Primero vamos a ver las reglas generales, desde donde se activan los distintos módulos:

\begin{itemize}
	\item \textbf{nuevo\_registro}: Registra un nuevo valor detectado por los sensores, añadiendo un hecho (valor\_registrado Tiempo
	Tipo Planta Valor) y un (ultimo\_registro Tipo Planta Tiempo).
	\item \textbf{ultimo\_reg}: Elimina de la base del conocimiento el anterior ultimo\_registro y se queda con el más reciente.
	\item \textbf{eliminarEsperarHumedad}: Esta regla elimina el valor de (esperar\_humedad Planta Tiempo) que había en el
	conocimiento al llegarle un nuevo valor de humedad mediante el sensor correspondiente. El hecho de esperar\_humedad indica que
	la planta ha sido regada recientemente, y por tanto, que el sistema no necesita deducir si debe o no ser regada de nuevo.
	\item \textbf{siguienteHora}: Esta regla permite adelantar la hora actual en 1 deduciendo cuál es la siguiente hora. Para
	activarla, se necesita un hecho que sea (siguiente\_hora) en la base del conocimiento.
	\item \textbf{calcularNoLluvia}: Regla que deduce si no se va a llover durante una determinada hora que viene dada por el hecho
	(prevision\_lluvia Hora Intensidad). Se considera que no va a llover si la intensidad está en el rango $[0, 0.2)$. Actualiza el
	conocimiento añadiendo un hecho que es (LluviaPrevista Hora no Intensidad).
	\item \textbf{calcularLluviaDebil}: Regla que deduce si la lluvia prevista a una hora dada por el hecho (prevision\_lluvia Hora
	Intensidad) es débil, es decir, si la intensidad está en el rango $[0.2, 6.5)$. Actualiza el conocimiento añadiendo un hecho que
	es (LluviaPrevista Hora debil Intensidad).
	\item \textbf{calcularLluviaMedio}: Regla que deduce si la lluvia prevista a una hora dada por el hecho (prevision\_lluvia Hora
	Intensidad) es moderada o media, es decir, si la intensidad está en el rango $[6.5, 15)$. Actualiza el conocimiento añadiendo un
	hecho que es (LluviaPrevista Hora moderada Intensidad).
	\item \textbf{calcularLluviaFuerte}: Regla que deduce si la lluvia prevista a una hora dada por el hecho (prevision\_lluvia Hora
	Intensidad) es fuerte, es decir, si la intensidad está en el rango $[15, 100)$. Actualiza el conocimiento añadiendo un
	hecho que es (LluviaPrevista Hora fuerte Intensidad).
	\item \textbf{calcularLluviaTorrencial}: Regla que deduce si la lluvia prevista a una hora dada por el hecho (prevision\_lluvia
	Hora Intensidad) es torrencial, es decir, si la intensidad es de 100 o más. Actualiza el conocimiento añadiendo un
	hecho que es (LluviaPrevista Hora torrencial Intensidad).
	\item \textbf{modificarPrediccionLluvia}: Regla que permite modificar el conocimiento asociado a las predicciones de tiempo
	cuando se detecta que en la base de conocimiento hay un hecho que es (modificar\_prevision Hora NuevaIntensidad), actualizando
	el conocimiento del sistema sobre la predicción de lluvia a esa hora de la forma correspondiente.
	\item \textbf{comenzarLluvia}: Regla que sirve para deducir que en la hora actual va a llover según la predicción que había para
	la hora actal. Se activa al cambiar la hora. Elimina la previsión que había para la nueva hora y añade a la base del conocimiento
	un hecho de tipo (LluviaActual Tipo), donde Tipo es débil, moderada, fuerte o torrencial.
	\item \textbf{pararLluvia}: Regla que deduce que la lluvia ha parado o que no llueve durante la hora actual de forma parecida
	a la que se hace en la regla anterior.
	\item \textbf{calcularRegarPlanta}: Regla que deduce si se debe regar o no la planta utilizando la información de los sensores
	y pasándosela al sistema difuso. Luego activa el módulo \textbf{CalculoSiguienteLluvia} para deducir si en las próximas 8
	horas se producirá lluvia o no. No se puede regar si se está vaporizando o está lloviendo.
	\item \textbf{calcularVaporizarPlanta}: Regla que deduce si se debe vaporizar la planta a partir de los valores de temperaturas
	que han detectado los sensores. Se encarga de activar el módulo \textbf{Vaporizacion}. No se puede vaporizar si se está regando
	o está lloviendo.
\end{itemize}

\subsubsection{Módulo Riego}

A continuación se describirán las reglas que utiliza el módulo Riego:

\begin{itemize}
	\item \textbf{riegoNo}: Regla que muestra información al usuario sobre lo que ha determinado el sistema de no regar la
	la planta. Además de eso, sale del módulo Riego, ya que no se va a regar la planta.
	\item \textbf{modificarRiegoLluviaDebil}: Regla para modificar la intensidad del riego si se ha deducido que en las próximas
	8 horas se va a producir una lluvia débil. En este caso, se ha supuesto que la lluvia débil no va a modificar en mucha cantidad
	la humedad de la planta, así que se deja la intensidad del riego se deja tal cuál. Sirve más para ofrecer información al usuario
	sobre las decisiones que va tomando el sistema en función de lo que va deduciendo.
	\item \textbf{modificarRiegoAltoLluviaModerada}: Regla para modificar la intensidad del riego si se ha deducido que en las
	próximas 8 horas se va a producir una lluvia moderada. En este caso, si el riego que se iba a hacer un riego de alta intensidad
	se reduce a uno de media, ya que la lluvia aportará algo de humedad. Todos los demás casos se dejan igual.
	\item \textbf{modificarRiegoAltoLluviaFuerte}: Regla para modificar la intensidad del riego si se ha deducido que en las próximas
	8 horas se va a producir una lluvia fuerte. Si se deduce que se producirá una lluvia fuerte y se tiene que el sistema difuso
	había determinado que se tiene que regar con intensidad fuerte, se reduce a débil, ya que la lluvia tendrá un gran aporte de
	humedad.
	\item \textbf{modificarRiegoMedioLluviaFuerte}: Regla para modificar la intensidad del riego si se ha deducido que en las
	próximas 8 horas se va a producir una lluvia fuerte. Si se deduce que se producirá una lluvia fuerte y se tiene que el sistema
	difuso había determinado que se tiene que regar con intensidad media, se reduce a débil, ya que la lluvia tendrá un gran aporte
	de humedad.
	\item \textbf{modificarRiegoLluviaTorrencialNo}: Regla para modificar la intensidad del riego si se ha deducido que en las
	próximas 8 horas se va a producir una lluvia torrencial. Si el sistema deduce que la humedad no está por debajo del nivel
	crítico corregirá el conocimiento eliminando las deducciones que había hecho sobre regar las plantas y saldrá del módulo Riego,
	ya que no se va a regar la planta. Esto se debe a que el sistema no considera necesario regar una planta a menos que su humedad
	esté por debajo del nivel crítico si ha detectado que se va a producir una lluvia torrencial.
	\item \textbf{modificarRiegoLluviaTorrencialBajo}: Regla para modificar la intensidad del riego si se ha deducido que en las
	próximas 8 horas se va a producir una lluvia torrencial. Si el sistema deduce que la humedad está por debajo del nivel
	crítico eliminará la intensidad del riego deducida anteriormente y determinará que se debe realizar un riego de baja intensidad,
	ya que la lluvia torrencial hará un gran aporte de humedad y no será necesario regar más.
	\item \textbf{riegoBajo}: Regla que, una vez deducido que el tipo de riego es de intensidad baja, activa el riego de
	intensidad baja con la humedad objetivo que ha determinado el sistema difuso (un poco por encima del mínimo).
	\item \textbf{riegoMedio}: Regla que, una vez deducido que el tipo de riego es de intensidad media, activa el riego de
	intensidad media con la humedad objetivo que ha determinado el sistema difuso (un poco por debajo del máximo).
	\item \textbf{riegoAlto}: Regla que, una vez deducido que el tipo de riego es de intensidad alta, activa el riego de
	intensidad alta con la humedad objetivo que ha determinado el sistema difuso (un poco por encima del máximo).
	\item \textbf{regarPlanta}: Regla para regar la planta. Cuando se ha decidido regar la planta, se aplica esta regla para ir
	regando la planta y aumentando su humedad hasta llegar al valor objetivo de humedad.
	\item \textbf{detenerRegarPlanta}: Regla para detener el regado de la planta una vez que se ha llegado al valor
	objetivo, desactivando por tanto el módulo Riego.
\end{itemize}

\subsubsection{Módulo CalculoSiguienteLluvia}

A continuación se describirán las reglas que utiliza el módulo CalculoSiguienteLluvia:

\begin{itemize}
	\item \textbf{contador}: Regla contador. Cuando se actibe este módulo y el sistema inserte el hecho (Contador HoraContador
	ValorContador) se activa esta regla y se empiezan a recorrer las previsiones en busca de la primera lluvia hasta que el
	ValorContador llegue a 0, incrementando cada vez en 1 la HoraContador.
	\item \textbf{contadorSiguienteLluvia}: Regla que se dispara cuando el contador se consigue llegar a una hora en la que
	hay lluvia. Añade el hecho (SiguienteLluvia Tipo Hora) a la base del conocimiento, donde Tipo se deduce segúna la previsión de
	lluvia que haya a esa hora,  y sale del módulo \textbf{CalculoSiguienteLluvia}, activando ahora el módulo \textbf{Riego}.
	\item \textbf{contadorNoLluvia}: Regla que se dispara cuando el contador llega a 0. Se deduce que no se va a producir ninguna
	lluvia ya que no se ha encontrado ninguna previsión de lluvia utilizando el contador. Por tanto, añade el hecho deducido
	(SiguienteLluvia no Hora), desactiva el módulo \textbf{CalculoSiguienteLluvia} y activa el módulo \textbf{Riego}.
\end{itemize}

\subsubsection{Módulo Vaporizacion}

A continuación se describirán las reglas que utiliza el módulo Vaporizacion:

\begin{itemize}
	\item \textbf{vaporizarPlanta}: Regla para vaporizar la planta. Una vez activo el módulo, se aplica esta regla para ir
	vaporizando la planta y reduciendo su temperatura hasta llegar al valor objetivo de temperatura.
	\item \textbf{detenerVaporizadoPlanta}: Regla para detener la vaporización de la planta una vez que se ha llegado al valor
	objetivo, desactivando por tanto el módulo Vaporizacion.
\end{itemize}

\newpage

\section{Manual de uso del sistema}

En esta sección se va a explicar brevemente cómo trabajar con el sistema.

Lo primero de todo es, una vez inicializado CLIPS, cargar el sistema. Opcionalmente se pueden cargar los valores iniciales de
los sensores y las previsiones del tiempo. Si no se quiere hacer, se tienen que introducir manualmente. Sin embargo, si se toma
este segundo curso de acción, se tienen que introducir, al menos, previsiones del tiempo para las 8 siguientes horas.

A continuación se puede ver un ejemplo de cómo se podrían cargar todos los archivos. Si no se quiere alguno de ellos, basta con
simplemente no introducirlo:

\begin{lstlisting}
(load sistema.clp)
(load sensores-init.clp)
(load previsiones.clp)
(reset)
(run)
\end{lstlisting}

Una vez cargados el sistema y los datos, se pueden insertar valores de los sensores de la siguiente forma:

\begin{lstlisting}
(assert (valor Tipo Planta ValorSensor))
\end{lstlisting}

\noindent donde \textit{Tipo} es Temperatura, Luminosidad o Humedad; \textit{Planta} es alguna de las plantas y \textit{ValorSensor}
es el valor medido por el sensor. Un ejemplo de esto sería el siguiente (el (run) se utiliza para insertar los hechos en la base
del conocimiento):

\begin{lstlisting}
(assert (valor Temperatura Tulipan 24))
(run)
\end{lstlisting}

Para ir insertando previsiones del tiempo, se puede hacer de la siguiente forma:

\begin{lstlisting}
(assert (prevision_lluvia Hora Intensidad))
\end{lstlisting}

\noindent donde \textit{Hora} es un valor en el rango $[0, 23]$ e \textit{Intensidad} es un valor real que representa la intensidad
de la lluvia en mm/h. Un ejemplo seria el siguiente:

\begin{lstlisting}
(assert (prevision_lluvia 11 45))
(run)
\end{lstlisting}

En caso de que el usuario se haya equivocado o la previsión del tiempo para una determinada hora cambie, se puede hacer lo siguiente
para modificar la previsión:


\begin{lstlisting}
(assert (modificar_prevision Hora NuevaIntensidad))
\end{lstlisting}

\noindent donde \textit{Hora} y \textit{NuevaIntensidad} tienen los mismos significados que antes. Por ejemplo, si quieremos
modificar la previsión que anteriormente teníamos de las 11 para que tenga una intensidad de 14, debemos hacer los siguiente:

\begin{lstlisting}
(assert (modificar_prevision 11 14))
(run)
\end{lstlisting}

El sistema también permite pasar las horas. Para hacerlo, basta con simplemente hacer lo siguiente:

\begin{lstlisting}
(assert (siguiente_hora))
(run)
\end{lstlisting}

Si se necesita modificar algo del sistema difuso, se recomienda echar un vistazo a este ejemplo de sistema difuso
\cite{bib:fuzzy-example} y a la documentación de las funciones implementadas \cite{bib:skfuzzy-tutorial}.

\newpage

\begin{thebibliography}{5}

\bibitem{bib:numpy}
NumPy
\\\url{https://www.numpy.org/}

\bibitem{bib:skfuzzy}
GitHub. \textit{scikit-fuzzy}
\\\url{https://github.com/scikit-fuzzy/scikit-fuzzy}

\bibitem{bib:fuzzy-example}
Ejemplo de sistema difuso
\\\url{https://pythonhosted.org/scikit-fuzzy/auto_examples/plot_tipping_problem.html#example-plot-tipping-problem-py}

\bibitem{bib:skfuzzy-tutorial}
Manual de las funciones utilizadas de \textit{scikit-fuzzy}
\\\url{https://pythonhosted.org/scikit-fuzzy/api/skfuzzy.html}


\end{thebibliography}

\end{document}

